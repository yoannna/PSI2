% !TEX program = XeLaTeX
\documentclass{VUMIFPSkursinis}
    \usepackage{algorithmicx}
    \usepackage{algorithm}
    \usepackage{algpseudocode}
    \usepackage{amsfonts}
    \usepackage{amsmath}
    \usepackage{bm}
    \usepackage{caption}
    \usepackage{color}
    \usepackage{enumitem}
    \usepackage{float}
    \usepackage{graphicx}
    \usepackage{listings}
    \usepackage{subfig}
    \usepackage{wrapfig}
    % Titulinio aprašas
    \university{Vilniaus universitetas}
    \faculty{Matematikos ir informatikos fakultetas}
    \department{Programų sistemų katedra}
    \papertype{Laboratorinis darbas}
    \title{Skrydžių bilietų paieškos sistema}
    \titleineng{Plane tickets search system}
    \status{2 kurso 4 grupės studentai}
    \author{Vardenis Pavardenis}
    \secondauthor{Vardenis Pavardenis}
    \thirdauthor{Vardenis Pavardenis}
    \fourthauthor{Vardenis Pavardenis}
    \date{Vilnius – \the\year}
    
    % Nustatymai
    % \setmainfont{Palemonas}   % Pakeisti teksto šriftą į Palemonas (turi būti įdiegtas sistemoje)
    \bibliography{bibliografija}
    
    \begin{document}
    \maketitle
      
        \tableofcontents
      
        \sectionnonum{Įvadas}
      
        \section{Reikalavimai}
            \subsection{Funkciniai reikalavimai}
                \begin{enumerate}[label=\textbf{FR\arabic*}.]
                    \item Funkcinis...
                \end{enumerate}
            \subsection{Nefunkciniai reikalavimai}
                \begin{enumerate}[label=\textbf{NFR\arabic*}.]
                    \item Turi būti palaikoma operacinė sistema su viena iš naršyklių:
                    \begin{itemize}
                        \item Mozilla Firefox (nuo 58 versijos)
                        \item Google Chrome (nuo 64 versijos)
                        \item Microsoft Internet Explorer (nuo 11 versijos)
                        \item Microsoft Edge (nuo 41 versijos)
                        \item Apple Safari (nuo 11 versijos)
                    \end{itemize}
                    \item Turi būti palaikomas HTTPS standartas.
                    \item Programų sistema turi būti susieta su avia kompanijų bilietų pirkimo svetaine.
                    \item Programų sistema turi turėti prieigą prie duomenų bazes, kurioje saugomi tvarkaraščiai, naudotojai, įvykiai ir užrašai.
                    \item Programų sistema turi būti sukurta naudojant \textit{Angular 5}.
                \end{enumerate}
      
        \section{Struktūrinis dalykinės srities modelis}
            \subsection{Esybių diagrama}
    
            \subsection{Žodynas}
                \begin{enumerate}[label=\textbf{E\arabic*}.]
                    \item Vartotojas - asmuo, norintis nusipirkti skrydžio bilietą.
                    \item Skrydžių bendrovė  - įmonė, kuri suteikia skrydžio paslaugą.
                \end{enumerate}
    
                \begin{table}[H]\footnotesize
                \centering
                \caption{Dalykinės srities metaforų reikalavimai}
                    {\begin{tabular}{|l|l|} \hline
                        Objektas & Metafora \\
                        \hline
                        \hline
                        Paieška & Skrydžių bilietų radimas pagal įvestus paieškos kriterijus \\
                        \hline
                        Rūšiavimas & Maršruto pasirinkimas pagal norimą kriterijų: greitis, kaina, greičio ir kainos santykis \\
                        \hline
                        Filtravimas & Rezultatų pateikimas pagal norimus kriterijus: persėdimų skaičius, skrydžių bendrovės \\
                        \hline
                        Kainų žirklės & Kainos amplitudė \\
                        \hline
                        Užsakymų istorija & Įsigytų skrydžių bilietų peržiūra \\
                        \hline
                        Skrydžių informacija & Įsigytos kelionės duomenys: data, laikas, oro uostai \\
                        \hline
                        Paieškos rezultatai & Gautos užklausos rezultatai \\
                        \hline
                    \end{tabular}}
                \label{tab:table example}
                \end{table}
    
            \subsection{Reikalavimų - struktūrinio dalykinės srities modelio atsekamumo matrica}
      
        \section{Užduotys}
            \subsection{Užduočių aprašymai}
            
                \begin{enumerate}[label=\textbf{U\arabic*}.]

                    \item \textbf{Ieškoti skrydžių}\\
                    Scenarijus
                    \\\textbf{Alternatyvūs scenarijai:}
                    \begin{itemize}
                        \item Scenarijus
                    \end{itemize}

                    \item \textbf{Rušiuoti paieškos rezultatus}\\
                    Scenarijus
                    \\\textbf{Alternatyvūs scenarijai:}
                    \begin{itemize}
                        \item Scenarijus
                    \end{itemize}

                    \item \textbf{Filtruoti paieškos rezultatus}\\
                    Scenarijus
                    \\\textbf{Alternatyvūs scenarijai:}
                    \begin{itemize}
                        \item Scenarijus
                    \end{itemize}

                    \item \textbf{Peržiūrėti detalią skrydžio informaciją}\\
                    Scenarijus
                    \\\textbf{Alternatyvūs scenarijai:}
                    \begin{itemize}
                        \item Scenarijus
                    \end{itemize}

                    \item \textbf{Pereiti į bilieto pardavėjo svetainę}\\
                    Scenarijus
                    \\\textbf{Alternatyvūs scenarijai:}
                    \begin{itemize}
                        \item Scenarijus
                    \end{itemize}

                    \item \textbf{Peržiūrėti būsimų skrydžių sąrašą}\\
                    Scenarijus
                    \\\textbf{Alternatyvūs scenarijai:}
                    \begin{itemize}
                        \item Scenarijus
                    \end{itemize}
                    
                \end{enumerate}
      
            \subsection{Reikalavimų - užduočių atsekamumo matrica}
      
        \sectionnonum{Išvados}
      
        \sectionnonum{Šaltiniai}
    
        \appendix  % Priedai
        % Prieduose gali būti pateikiama pagalbinė, ypač darbo autoriaus savarankiškai
        % parengta, medžiaga. Savarankiški priedai gali būti pateikiami ir
        % Priedai taip pat numeruojami ir vadinami. Darbo tekstas
        % su priedais susiejamas nuorodomis.
      
    \end{document}
      