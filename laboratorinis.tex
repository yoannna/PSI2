% !TEX program = XeLaTeX
\documentclass{VUMIFPSkursinis}
\usepackage{algorithmicx}
\usepackage{algorithm}
\usepackage{algpseudocode}
\usepackage{amsfonts}
\usepackage{amsmath}
\usepackage{bm}
\usepackage{caption}
\usepackage{color}
\usepackage{enumitem}
\usepackage{float}
\usepackage{graphicx}
\usepackage{listings}
\usepackage{subfig}
\usepackage{wrapfig}
% Titulinio aprašas
\university{Vilniaus universitetas}
\faculty{Matematikos ir informatikos fakultetas}
\department{Programų sistemų katedra}
\papertype{Laboratorinis darbas}
\title{Skrydžių bilietų paieškos sistema}
\titleineng{Plane tickets search system}
\status{2 kurso 4 grupės studentai}
\author{Vardenis Pavardenis}
\secondauthor{Vardenis Pavardenis}
\thirdauthor{Vardenis Pavardenis}
\fourthauthor{Vardenis Pavardenis}
\date{Vilnius – \the\year}

% Nustatymai
% \setmainfont{Palemonas}   % Pakeisti teksto šriftą į Palemonas (turi būti įdiegtas sistemoje)
\bibliography{bibliografija}

\begin{document}
\maketitle
  
    \tableofcontents
  
    \sectionnonum{Įvadas}
  
    \section{Reikalavimai}
        \subsection{Funkciniai reikalavimai}
            \begin{enumerate}[label=FR\arabic*.]
                \item Funkcinis...
            \end{enumerate}
        \subsection{Nefunkciniai reikalavimai}
            \begin{enumerate}[label=NFR\arabic*.]
                \item Neunkcinis...
            \end{enumerate}
  
    \section{Struktūrinis dalykinės srities modelis}
        \subsection{Esybių diagrama}

        \subsection{Žodynas}

            \begin{table}[H]\footnotesize
            \centering
            \caption{Dalykinės srities metaforų reikalavimai}
                {\begin{tabular}{|l|l|} \hline
                    Objektas & Metafora \\
                    \hline
                    \hline
                    Vartotojas  & Asmuo, norintis nusipirkti skrydžio bilietą \\
                    \hline
                    Skrydžių bendrovė  & Įmonė, kuri suteikia skrydžio paslaugą \\
                    \hline
                    Paieška & Skrydžių bilietų radimas pagal įvestus paieškos kriterijus \\
                    \hline
                    Rūšiavimas & Maršruto pasirinkimas pagal norimą kriterijų: greitis, kaina, greičio ir kainos santykis \\
                    \hline
                    Filtravimas & Rezultatų pateikimas pagal norimus kriterijus: persėdimų skaičius, skrydžių bendrovės \\
                    \hline
                    Kainų žirklės & Kainos amplitudė \\
                    \hline
                    Užsakymų istorija & Įsigytų skrydžių bilietų peržiūra \\
                    \hline
                    Skrydžių informacija & Įsigytos kelionės duomenys: data, laikas, oro uostai \\
                    \hline
                    Paieškos rezultatai & Gautos užklausos rezultatai \\
                    \hline
                \end{tabular}}
            \label{tab:table example}
            \end{table}

        \subsection{Reikalavimų - struktūrinio dalykinės srities modelio atsekamumo matrica}
  
    \section{Užduotys}
        \subsection{Užduočių aprašymai}
        
            \begin{enumerate}[label=U\arabic*.]
                \item Datos įvedimas
                \item Kainos įvedimas
                \item Miestų įvedimas
                \item Esamos informacijos peržiūra
                \item Paieškos rezultatų filtravimas
                \item Paieškos rezultatų rūšiavimas
            \end{enumerate}
  
        \subsection{Reikalavimų - užduočių atsekamumo matrica}
  
    \sectionnonum{Išvados}
  
    \sectionnonum{Šaltiniai}

    \appendix  % Priedai
    % Prieduose gali būti pateikiama pagalbinė, ypač darbo autoriaus savarankiškai
    % parengta, medžiaga. Savarankiški priedai gali būti pateikiami ir
    % Priedai taip pat numeruojami ir vadinami. Darbo tekstas
    % su priedais susiejamas nuorodomis.
  
\end{document}
  